\documentclass{article}

\usepackage{color}
\definecolor{dkgreen}{rgb}{0,0.6,0}
\definecolor{gray}{rgb}{0.5,0.5,0.5}
\definecolor{mauve}{rgb}{0.58,0,0.82}
\usepackage[margin=1in]{geometry}
\usepackage{fancyhdr}
\pagestyle{fancy}
\lhead{\today}
\chead{Ying Qiao SID:21412301}
\rhead{Stat215B Sp13: Assignment 2}
\lfoot{}
\cfoot{\thepage}
\rfoot{}
\usepackage{graphicx}
\usepackage{textcomp}
\usepackage{lmodern}
\usepackage[T1]{fontenc}
\usepackage{listings}

%% for inline R code: if the inline code is not correctly parsed, you will see a message
\newcommand{\rinline}[1]{SOMETHING WORNG WITH knitr}
%% begin.rcode setup, include=FALSE
% opts_chunk$set(fig.path='figure/latex-', cache.path='cache/latex-')
%% end.rcode


\begin{document}
\section*{Models for time to failure}

\hspace{12 pt} With the hazard function defined below, \newline
$\displaystyle H(t) = \frac{f(t)}{1-F(t)}$, \newline
$F(t)$ and $f(t)$ are the CDF and PDF respectively. 

\vspace{6 pt}
If the exponential distribution is used for the model, i.e., \newline
$f(t) = \lambda exp(-\lambda t)$ and $F(t) = 1 - exp(-\lambda t)$.\newline
The modeled harzard function will be \newline
$H(t) = \lambda$, \newline
which is a constant function and is therefore unreasonable for
describing \newline
$H(t) = \lim_{\epsilon \to 0} \frac{1}{\epsilon}$P(Get disease during
time $t$ to $t+\epsilon$ | Never had disease up tp time $t$).

\vspace{6 pt}
So, one alternative is to model with \textit{Weibull} distribution,
given the PDF and CDF below,\newline
$\displaystyle g_{\alpha ,\beta}(t)=\frac{\beta}{\alpha}(\frac{t}{\alpha})^{\beta-1}exp(-(\frac{t}{\alpha})^{\beta})$ and
$\displaystyle G_{\alpha ,\beta}(t)=1 - exp(-(\frac{t}{\alpha})^{\beta})$. \newline
We will have the harzard function, \newline
$\displaystyle H(t)=\frac{\beta}{\alpha}(\frac{t}{\alpha})^{\beta-1}$. \newline
This is a more useful model compared to the one generated by
exponential distribution. In this case, with appropriate parameters
$\alpha > 0, \beta > 0$, we may choose different scenarios for
different studies. For example, $\alpha=2, \beta=2$ will give us
positive correlation in \textit{getting disease at time t} and
\textit{no disease up till time t}, which is a plausible assumption.



\newpage
\section*{Survival curves}
\hspace{12 pt} A new surgical procedure effectiveness test is set up
as the following: \newline
$N = 1000$ individuals divided equally into
surgical treament group $X$ and control group $Y$; \newline
times to death are \textit{i.i.d} with $X_i \sim G_{3,2}$ and $Y_i\sim G_{2,2}$. \newline  
We would like to estimate the two survival functions
$S_X(t) = P(X_i > t)$ and $S_Y(t) = P(Y_i > t)$.


\subsubsection*{1. Random generation for the Weibull distribution}
\hspace{12 pt} The R function that I write for drawing
\textit{Weibull} random variables employs the inverse CDF sampling
method. From the CDF, we derive that the inverser CDF is 
$G^{-1}_{\alpha , \beta}(x) = \alpha(-\log(1-x))^{1/\beta}$.

%%  begin.rcode ass2-1, cache=TRUE,results="markup",message=FALSE
%%my.rweibull <- function (n, shape, scale = 1){
%%  if (n <= 0 || shape <= 0 || scale <= 0){
%%    warning("Invalid arguments! All arguments should be positive.")
%%    return ("NaN")
%%  }
%%  u <- runif(n)
%%  x <- scale * (-log(1 - u))^(1/shape)
%%  return (x)
%%}
%%  end.rcode

\subsubsection*{2. Kaplan-Meier survival-function estimate}
\hspace{12 pt} With the two input vectors, \textit{times} of event
times and \textit{censor} of censoring indicators, we can produce the
Kaplan-Meier survivial-function estimate from \newline
$\displaystyle S(t_i) = \frac{N(t_i)-E(t_i)}{N(t_i)} \times S(t_{i-1})$, 
with $S(0) = 1$ \newline
where vector $t$ contains valid estimation points (excluding censored
times), $N$ denotes the number of subjects at risk and $E$ represents
the number of events(deaths) at the evaluated time. The output of my
function contains the estimation points $t$ vector, survival rates
$surv$ vector and the lookup table function for the Kaplan-Meier estimates.


%%  begin.rcode ass2-2, cache=TRUE, results="markup", message=FALSE
%%my.KMsurv <- function (times, censor){
%%  if (length(times) <= 0 || length(censor) <= 0)
%%    stop("Invalid input vectors!")
%%  if (length(times) != length(censor))
%%    stop("Mismatch of dimension in input vectors!")
%%  t <- c(0, sort(unique(times[censor]))) # prepare t estimates points
%%  s <- rep(0, length(t)) # create empty survival time estimates vector
%%  s[1] <- 1 # initialize 
%%  for (i in 2:length(t)){
%%    n <- length(times[times >= t[i]]) # num of subjects at risk at time t
%%    e <- length(times[times == t[i]]) # num of events at time t
%%    s[i] <- (n-e)/n * s[i-1]
%%  }
%%  # estimated survival function S(t)
%%  my.surv <- function (eval){
%%    return (s[t == eval])
%%  }
%%  # return a list of related KM objects
%%  km <- list(t = t, surv = s, s.func = my.surv)
%%  return (km)
%%}
%%  end.rcode

\subsubsection*{3. Simple simulations with constant cut-off}
\hspace{12 pt} Since the study has a length of 5 years, we set the
censoring indicator for the data by cut-off at 5. We then
simulate the performace of the clinic trial with the setup described
above. The graphical comparison for both treament and control groups
between the true and estimated survival curves are shown in Figure 1.


We can see that the estimated curves are both very close to the real curves
before the cut-off point, but beyond that point (in this case, 5), we
have no estimation at all. Moreover, the treatment group shows higher
survival rate from the plots indicating effectiveness of this surgical procedure.


%%  begin.rcode ass2-3, cache=TRUE, echo=FALSE, results="markup",dev='pdf',fig.align="center",fig.height=4.5,fig.width=7,fig.cap="Visual comparisons for constant cut-off simulations"
%%set.seed(0)
%%X <- my.rweibull(500,shape=2,scale=3)
%%set.seed(100)
%%Y <- rweibull(500,shape=2,scale=2)
%%cen <- rep(TRUE, 500)
%%# true curve
%%x.km0 <- my.KMsurv(X, cen)
%%y.km0 <- my.KMsurv(Y, cen)
%%# 3. simple simulation censoring: cut-off at 5 years
%%cen.x <- X <= 5
%%cen.y <- Y <= 5
%%x.km1 <- my.KMsurv(X, cen.x)
%%y.km1 <- my.KMsurv(Y, cen.y)
%%# visual comparisons
%%par(mfrow=c(1,2),cex.main=1)
%%typ='s'; lwd=2; col0='blue';lty0=1;col='red';lty=4
%%xlab="Time (year)"; ylab="Survival Rate"; ylim=c(0,1)
%%plot(x.km0$t, x.km0$surv, type=typ,col=col0,lwd=lwd,lty=lty0, xlim=c(0,max(X)),ylim=ylim, xlab=xlab,ylab=ylab, main="Survival Curves for Treatment Group X")
%%lines(x.km1$t, x.km1$surv, type=typ,col=col,lwd=lwd,lty=lty)
%%legend("topright",c("True","Estimated"),col=c(col0,col),lty=c(lty0,lty))
%%plot(y.km0$t, y.km0$surv, type=typ,col=col0,lwd=lwd,lty=lty0, xlim=c(0,max(X)),ylim=ylim, xlab=xlab,ylab=ylab, main="Survival Curves for Control Group Y")
%%lines(y.km1$t, y.km1$surv, type=typ,col=col,lwd=lwd,lty=lty)
%%legend("topright",c("True","Estimated"),col=c(col0,col),lty=c(lty0,lty))
%%  end.rcode

\subsubsection*{4. Advanced simulations with independent random cut-off}
\hspace{12 pt} We try to inject randomness into the censoring for
estimation beyond the clinical cut-off point. Here, we assume
\textit{i.i.d} $Z_i \sim ~ Exp(\lambda = 1/10)$, that gives the time
at which $i$ will be censored. $Z$ being Independent of $X$ and $Y$, our
simulation results of the trial setup are shown graphically in Figure 2.


From the plot, we can see that we do have more estimations beyond the
clinical trial cut-off point, which makes the estimated curve more
complete. However, we have seen more error in this plot. The estimated
curve for the treatment group seems to overestimate the survival
rates. This is probably caused by the fact that we are adding more
\textit{effective} patients into the counts, making the estimations
overly optimistic.

%%  begin.rcode ass2-4, cache=TRUE, echo=FALSE,results="markup",dev='pdf',fig.align="center",fig.height=4.5,fig.width=7,fig.cap="Visual comparisons for independent random cut-off simulations"
%%set.seed(0);   Z <- rexp(500, rate=1/10)
%%cen.x <- X <= Z
%%set.seed(100); Z <- rexp(500, rate=1/10)
%%cen.y <- Y <= Z
%%x.km2 <- my.KMsurv(X, cen.x)
%%y.km2 <- my.KMsurv(Y, cen.y)
%%# visual comparisons
%%par(mfrow=c(1,2),cex.main=1)
%%plot(x.km0$t, x.km0$surv, type=typ,col=col0,lwd=lwd,lty=lty0, xlim=c(0,max(X)),ylim=ylim, xlab=xlab,ylab=ylab, main="Survival Curves for Treatment Group X")
%%lines(x.km2$t, x.km2$surv, type=typ,col=col,lwd=lwd,lty=lty)
%%legend("topright",c("True","Estimated with Z"),col=c(col0,col),lty=c(lty0,lty))
%%plot(y.km0$t, y.km0$surv, type=typ,col=col0,lwd=lwd,lty=lty0, xlim=c(0,max(X)),ylim=ylim, xlab=xlab,ylab=ylab, main="Survival Curves for Control Group Y")
%%lines(y.km2$t, y.km2$surv, type=typ,col=col,lwd=lwd,lty=lty)
%%legend("topright",c("True","Estimated with Z"),col=c(col0,col),lty=c(lty0,lty))
%%  end.rcode

\subsubsection*{5. Advanced simulations with dependent random cut-off}
\hspace{12 pt} A little tweak to 4, here we assume 
$Z_I|X_I < 2 \sim Exp(\lambda = 1/10)$ and $Z_i|X_i \geq 2 \sim Exp(\lambda = 1/5)$. \newline
This creates dependence of $Z$ on $X$, similarly for $Z$ and $Y$,
which coule arise in a study where the sicker patients are more likely
to remain under the care of their doctors and have longer expected
follow-up time. The results of this run of simulation are shown in
Figure 3.

Similar to Figure 2, we can see a even more complete estimated curve
but also more optimistically biased curve, especially for the treament
group. This censoring scenario amplifies the effectiveness of the
treament on the sicker patients by prolonging their follow-up time
threshold and adding more effectively treated patients into the
counts. This seems overly optimistic for the estimation of the
survival rates.

%%  begin.rcode ass2-5, cache=TRUE, echo=FALSE,results="markup",dev='pdf',fig.align="center",fig.height=4.5,fig.width=7,fig.cap="Visual comparisons for dependent random cut-off simulations"
%%Z <- rep(0, 500)
%%cut.x <- X < 2; n.l2x <- length(X[cut.x])
%%set.seed(0);   Z[cut.x]  <- rexp(n.l2x, rate=1/10)
%%set.seed(10);  Z[!cut.x] <- rexp(500-n.l2x, rate=1/5)
%%cen.x <- X <= Z
%%Z <- rep(0, 500)
%%cut.y <- Y < 2; n.l2y <- length(Y[cut.y])
%%set.seed(100); Z[cut.y]  <- rexp(n.l2y, rate=1/10) 
%%set.seed(110); Z[!cut.y] <- rexp(500-n.l2y, rate=1/5)
%%cen.y <- Y <= Z
%%x.km3 <- my.KMsurv(X, cen.x)
%%y.km3 <- my.KMsurv(Y, cen.y)
%%# visual comparisons
%%par(mfrow=c(1,2),cex.main=1)
%%plot(x.km0$t, x.km0$surv, type=typ,col=col0,lwd=lwd,lty=lty0, xlim=c(0,max(X)),ylim=ylim, xlab=xlab,ylab=ylab, main="Survival Curves for Treatment Group X")
%%lines(x.km3$t, x.km3$surv, type=typ,col=col,lwd=lwd,lty=lty)
%%legend("topright",c("True","Estimated with Z|X"),col=c(col0,col),lty=c(lty0,lty))
%%plot(y.km0$t, y.km0$surv, type=typ,col=col0,lwd=lwd,lty=lty0, xlim=c(0,max(X)),ylim=ylim, xlab=xlab,ylab=ylab, main="Survival Curves for Control Group Y")
%%lines(y.km3$t, y.km3$surv, type=typ,col=col,lwd=lwd,lty=lty)
%%legend("topright",c("True","Estimated with Z|Y"),col=c(col0,col),lty=c(lty0,lty))
%%  end.rcode


\end{document}
