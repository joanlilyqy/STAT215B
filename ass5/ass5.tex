\documentclass{article}

\usepackage{color}
\definecolor{dkgreen}{rgb}{0,0.6,0}
\definecolor{gray}{rgb}{0.5,0.5,0.5}
\definecolor{mauve}{rgb}{0.58,0,0.82}
\usepackage[margin=1in]{geometry}
\usepackage{fancyhdr}
\pagestyle{fancy}
\lhead{\today}
\chead{Ying Qiao SID:21412301}
\rhead{Stat215B Sp13: Assignment 4}
\lfoot{}
\cfoot{\thepage}
\rfoot{}
\usepackage{graphicx}
\usepackage{textcomp}
\usepackage{lmodern}
\usepackage[T1]{fontenc}
\usepackage{listings}
\usepackage{amssymb,amsmath}
\usepackage{bm}

%% for inline R code: if the inline code is not correctly parsed, you will see a message
\newcommand{\rinline}[1]{SOMETHING WORNG WITH knitr}
%% begin.rcode setup, include=FALSE
% opts_chunk$set(fig.path='figure/latex-', cache.path='cache/latex-')
%% end.rcode


\begin{document}
\section*{Math stats}
Efron (2010) exercises

\subsubsection*{1.1}
With the given distribution
\begin{displaymath}
\begin{split}
\mu \sim \mathcal{N}(0,A) \rightarrow  &
f(\mu) = \frac{1}{\sqrt{2 \pi A}} \exp\{-\frac{\mu^2}{2A}\}  \\
z|\mu \sim \mathcal{N}(0,1) \rightarrow & 
f(z|\mu) = \frac{1}{\sqrt{2 \pi}} \exp\{-\frac{(z-\mu)^2}{2}\}
\end{split}
\end{displaymath}
So, we have
\begin{displaymath}
\begin{split}
f(\mu)\cdot f(z|\mu) &= \frac{1}{2\pi \sqrt{A}} \exp \{-\frac{1}{2}
(\frac{A+1}{A}\mu^2 - 2\mu z + z^2)\} \\
& = \frac{1}{2\pi \sqrt{A}} \exp \{-\frac{1}{2B}
(\mu^2 - 2B\mu z + Bz^2)\} ;       B=\frac{A}{A+1} \\
& = \frac{1}{2\pi \sqrt{A}} \exp \{-\frac{1}{2} (\frac{\mu -  Bz}{\sqrt{B}})^2\} 
\cdot \exp \{-\frac{Az^2}{2(A+1)^2} \} 
\end{split}
\end{displaymath}
Then, the integral
\begin{displaymath}
\begin{split}
f(z) &= \int {f(\mu)f(z|\mu)} d\mu \\
& = \frac{1}{2\pi \sqrt{A}} \exp \{-\frac{Az^2}{2(A+1)^2} \} 
\cdot \sqrt{B} \int_{-\infty}^{+\infty}  
\exp \{-\frac{1}{2} (\frac{\mu -  Bz}{\sqrt{B}})^2\} 
d (\frac{\mu -  Bz}{\sqrt{B}}) \\
& = \frac{\sqrt{B}}{\sqrt{2\pi A}} \exp \{-\frac{Az^2}{2(A+1)^2} \} 
\end{split}
\end{displaymath}
Finally, we get
\begin{displaymath}
\begin{split}
f(\mu |z) &= \frac{f(\mu)\cdot f(z|\mu)}{f(z)} \\
& = \frac{1}{\sqrt{2\pi B}} \exp \{-\frac{1}{2} (\frac{\mu -  Bz}{\sqrt{B}})^2\} \\
\end{split}
\end{displaymath}
that is, 
\begin{displaymath}
\mu |z \sim \mathcal{N}(Bz, B) , B= A/(A+1)
\end{displaymath}


\subsubsection*{1.2}
(i) (1.17) \newline
We have Bayes estimator
\begin{displaymath}
\hat{\bm{\mu}}^{(Bayes)} = B\bm{z}; B = \frac{A}{A+1}
\end{displaymath}
Then, the risk function (total square error loss)
\begin{displaymath}
\begin{split}
R^{(Bayes)}(\bm{\mu}) &= E_{\bm{z|\mu}}\{||\hat{\bm{\mu}} - \bm{\mu}||^2\} \\
& = E_{\bm{z|\mu}}\{||B\bm{z} - B\bm{\mu} + B\bm{\mu} - \bm{\mu}||^2\} \\
& = B^2 E_{\bm{z|\mu}}\{||\bm{z} - \bm{\mu}||^2\} + (B-1)^2E_{\bm{z|\mu}}\{||\bm{\mu}||^2\}
\end{split}
\end{displaymath}
With the length of $N$ vector $\bm{z|\mu} \sim \mathcal{N}_N(\bm{\mu},\bm{I})$, we have finally got
\begin{displaymath}
\begin{split}
R^{(Bayes)}(\bm{\mu}) = B^2 N + (1-B)^2 ||\bm{\mu}||^2
\end{split}
\end{displaymath}
(ii) (1.18) \newline
With the length of $N$ vector $\bm{\mu} \sim \mathcal{N}_N(\bm{0},
A\bm{I})$, using (1.17), the overall Bayes risk
\begin{displaymath}
\begin{split}
R^{(Bayes)} & = E_{\bm{\mu}}\{ R^{(Bayes)}(\bm{\mu}) \} \\
& = B^2 N + (1-B)^2 E_{\bm{\mu}} \{||\bm{\mu}||^2\}; B=\frac{A}{A+1} \\
& = \frac{NA^2}{(A+1)^2} + \frac{1}{(A+1)^2} NA \\
& = \frac{NA}{A+1}
\end{split}
\end{displaymath}


\subsubsection*{1.4}
(a) (1.31) \newline
We have (1.30), that is
\begin{displaymath}
E_{\bm{z|\mu}} \{||\hat{\bm{\mu}} - \bm{\mu}||^2\} 
= E_{\bm{z|\mu}} \{||\bm{z} - \hat{\bm{\mu}}||^2\} - N 
+ 2\sum_{i=1}^N E_{\bm{z|\mu}}\{\frac{\partial \hat{\mu}_i}{\partial z_i}\}
\end{displaymath}
Then, with 
\begin{displaymath}
\begin{split}
\hat{\bm{\mu}}^{(JS)} &= (1 - \frac{N-2}{S}) \bm{z} \\
S &=||\bm{z}||^2= \sum_{i=1}^N z_i^2
\end{split}
\end{displaymath}
We get
\begin{displaymath}
\begin{split}
E_{\bm{z|\mu}} \{||\hat{\bm{\mu}}^{(JS)} - \bm{\mu}||^2\} 
& =  E_{\bm{z|\mu}} \{||\frac{N-2}{S} \bm{z}||^2\} - N + 2\sum_{i=1}^N
E_{\bm{z|\mu}} \{(1-\frac{N-2}{S}) + \frac{N-2}{S^2} 2z_i^2 \}  \\
& = E_{\bm{z|\mu}} \{(\frac{N-2}{S})^2 ||\bm{z}||^2\} - N + 
2 E_{\bm{z|\mu}} \{N(1-\frac{N-2}{S}) + \frac{2(N-2)}{S^2} \sum_{i=1}^N z_i^2\} \\
& = E_{\bm{z|\mu}} \{\frac{(N-2)^2}{S} \} - N + 2N -
2 E_{\bm{z|\mu}} \{\frac{N(N-2)}{S} - \frac{2(N-2)}{S} \} \\
& = N - E_{\bm{z|\mu}} \{\frac{(N-2)^2}{S} \}
\end{split}
\end{displaymath}
(b) (1.24) \newline
We have (1.31), that is shown above; and with marginal 
$\bm{z} \sim \mathcal{N}_N(\bm{0},(A+1)\bm{I})$, 
\begin{displaymath}
E\{ \frac{N-2}{S} \} = \frac{1}{A+1}
\end{displaymath}
So we can derive that
\begin{displaymath}
\begin{split}
R^{(JS)} & = E_{\bm{\mu}} \{ R^{(JS)} (\bm{\mu}) \} \\
& = E_{\bm{\mu}} \{ E_{\bm{z|\mu}} \{||\hat{\bm{\mu}}^{(JS)} - \bm{\mu}||^2\} \} \\
& = N - E_{\bm{\mu}} \{ E_{\bm{z|\mu}} \{ \frac{(N-2)^2}{S} \} \} \\
& = N - (N-2) E_{\bm{z}} \{ \frac{N-2}{S} \} \\
& = N - \frac{N-2}{A+1}  = N \frac{A}{A+1} + \frac{2}{A+1}
\end{split}
\end{displaymath}


\subsubsection*{1.5}
If we assume in Table 1.2, for $i = 1, ..., n$,
$\mu_i \sim \mathcal{N}(0, A)$, and we also have
$\bm{z|\mu} \sim \mathcal{N}_n(\bm{\mu},\bm{I})$, then we can use all the
conclusions from the hierachical Normal-Normal model. \newline
So based on the derivations in the book, the simulated total square
error of the James-Stein estimator should be close to the theoretical 
value, i.e. Bayes risk of JS estimator.
\begin{displaymath}
TSE \sim R^{(JS)} = (nA + 2)/(A+1)
\end{displaymath}
Here, we can estimate
\begin{displaymath}
\hat{A} = \frac{1}{n} \sum_{i=1}^{n=10} \mu_i^2
\end{displaymath}
%%  begin.rcode ass5-1-15,cache=TRUE,results="markup",message=FALSE,echo=FALSE
%%n <- 10
%%mu <- c(-.81, -.39, -.39, -.08, .69, .71, 1.28, 1.32, 1.89, 4.00)
%%A.hat <- sum(mu^2) / n
%%TSE <- (n*A.hat+2)/(A.hat+1)
%%  end.rcode
So, we get $\hat{A}=$ \rinline{A.hat} and $TSE=$ \rinline{TSE}, which is
close to the $8.13$ value in the table.


\newpage
\section*{Simulation}
\hspace{12 pt} In Monte-Carlo simulations (with runs),
we set the true parameter values to be 




\newpage
\section*{Shrinking radon}
\hspace{12 pt} In Monte-Carlo simulations (with runs),
we set the true parameter values to be 




\end{document}
