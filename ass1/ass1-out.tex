\documentclass{article}\usepackage{graphicx, color}
%% maxwidth is the original width if it is less than linewidth
%% otherwise use linewidth (to make sure the graphics do not exceed the margin)
\makeatletter
\def\maxwidth{ %
  \ifdim\Gin@nat@width>\linewidth
    \linewidth
  \else
    \Gin@nat@width
  \fi
}
\makeatother

\IfFileExists{upquote.sty}{\usepackage{upquote}}{}
\definecolor{fgcolor}{rgb}{0.2, 0.2, 0.2}
\newcommand{\hlnumber}[1]{\textcolor[rgb]{0,0,0}{#1}}%
\newcommand{\hlfunctioncall}[1]{\textcolor[rgb]{0.501960784313725,0,0.329411764705882}{\textbf{#1}}}%
\newcommand{\hlstring}[1]{\textcolor[rgb]{0.6,0.6,1}{#1}}%
\newcommand{\hlkeyword}[1]{\textcolor[rgb]{0,0,0}{\textbf{#1}}}%
\newcommand{\hlargument}[1]{\textcolor[rgb]{0.690196078431373,0.250980392156863,0.0196078431372549}{#1}}%
\newcommand{\hlcomment}[1]{\textcolor[rgb]{0.180392156862745,0.6,0.341176470588235}{#1}}%
\newcommand{\hlroxygencomment}[1]{\textcolor[rgb]{0.43921568627451,0.47843137254902,0.701960784313725}{#1}}%
\newcommand{\hlformalargs}[1]{\textcolor[rgb]{0.690196078431373,0.250980392156863,0.0196078431372549}{#1}}%
\newcommand{\hleqformalargs}[1]{\textcolor[rgb]{0.690196078431373,0.250980392156863,0.0196078431372549}{#1}}%
\newcommand{\hlassignement}[1]{\textcolor[rgb]{0,0,0}{\textbf{#1}}}%
\newcommand{\hlpackage}[1]{\textcolor[rgb]{0.588235294117647,0.709803921568627,0.145098039215686}{#1}}%
\newcommand{\hlslot}[1]{\textit{#1}}%
\newcommand{\hlsymbol}[1]{\textcolor[rgb]{0,0,0}{#1}}%
\newcommand{\hlprompt}[1]{\textcolor[rgb]{0.2,0.2,0.2}{#1}}%

\usepackage{framed}
\makeatletter
\newenvironment{kframe}{%
 \def\at@end@of@kframe{}%
 \ifinner\ifhmode%
  \def\at@end@of@kframe{\end{minipage}}%
  \begin{minipage}{\columnwidth}%
 \fi\fi%
 \def\FrameCommand##1{\hskip\@totalleftmargin \hskip-\fboxsep
 \colorbox{shadecolor}{##1}\hskip-\fboxsep
     % There is no \\@totalrightmargin, so:
     \hskip-\linewidth \hskip-\@totalleftmargin \hskip\columnwidth}%
 \MakeFramed {\advance\hsize-\width
   \@totalleftmargin\z@ \linewidth\hsize
   \@setminipage}}%
 {\par\unskip\endMakeFramed%
 \at@end@of@kframe}
\makeatother

\definecolor{shadecolor}{rgb}{.97, .97, .97}
\definecolor{messagecolor}{rgb}{0, 0, 0}
\definecolor{warningcolor}{rgb}{1, 0, 1}
\definecolor{errorcolor}{rgb}{1, 0, 0}
\newenvironment{knitrout}{}{} % an empty environment to be redefined in TeX

\usepackage{alltt}

\usepackage{color}
\definecolor{dkgreen}{rgb}{0,0.6,0}
\definecolor{gray}{rgb}{0.5,0.5,0.5}
\definecolor{mauve}{rgb}{0.58,0,0.82}
\usepackage[margin=1in]{geometry}
\usepackage{fancyhdr}
\pagestyle{fancy}
\lhead{\today}
\chead{Ying Qiao SID:21412301}
\rhead{Stat215B Sp13: Assignment 1}
\lfoot{}
\cfoot{\thepage}
\rfoot{}
\usepackage{graphicx}
\usepackage{textcomp}
\usepackage{lmodern}
\usepackage[T1]{fontenc}
\usepackage{listings}

%% for inline R code: if the inline code is not correctly parsed, you will see a message
\newcommand{\rinline}[1]{SOMETHING WORNG WITH knitr}




\begin{document}
\section*{Data Preparation}

\hspace{12 pt} Within \textit{babies.data}, there are 1236 observations of 7 
important variables related to the live births.
The data cleaning procedure is mainly focused on the missing value
removal and catagorical variable conversion. 
Among several variables, for example, \textit{gestation} indicating the days
of pregnancy, labels missing values with the number of 999. On the other hand,
\textit{parity} indicating whether the baby is the first-born, needs
to be coded categorically with description of levels. 
The initial cleaning results on the data frame \textit{babies} are
shown below.


\begin{knitrout}
\definecolor{shadecolor}{rgb}{0.969, 0.969, 0.969}\color{fgcolor}\begin{kframe}
\begin{verbatim}
##    bwt gestation parity age height weight smoke
## 1  120       284 not.FB  27     62    100    No
## 2  113       282 not.FB  33     64    135    No
## 3  128       279 not.FB  28     64    115   Yes
## 4  123        NA not.FB  36     69    190    No
## 5  108       282 not.FB  23     67    125   Yes
## 6  136       286 not.FB  25     62     93    No
## 7  138       244 not.FB  33     62    178    No
## 8  132       245 not.FB  23     65    140    No
## 9  120       289 not.FB  25     62    125    No
## 10 143       299 not.FB  30     66    136   Yes
\end{verbatim}
\end{kframe}
\end{knitrout}


When working on the statistical claims, we need to add serveral categorical
variables into the current data frame.
\textit{premature} is added as a two-level factor variable to indicate
whether a baby was born prematurely; and a premature birth is defined
as occuring prior to the 37th week ($37\times 7 = 259$ days) of
gestation. Similarly, \textit{f.height} and \textit{f.weight} are
added to divide the mothers into groups based on median height and
weight in the data. So the final clean data before any analysis is
sketched below.

\begin{knitrout}
\definecolor{shadecolor}{rgb}{0.969, 0.969, 0.969}\color{fgcolor}\begin{kframe}
\begin{verbatim}
##    bwt gestation parity age height weight smoke premature f.height  f.weight
## 1  120       284 not.FB  27     62    100    No    not.PM    short     light
## 2  113       282 not.FB  33     64    135    No    not.PM    short     heavy
## 3  128       279 not.FB  28     64    115   Yes    not.PM    short     light
## 4  123        NA not.FB  36     69    190    No      <NA>     tall     heavy
## 5  108       282 not.FB  23     67    125   Yes    not.PM     tall     light
## 6  136       286 not.FB  25     62     93    No    not.PM    short     light
## 7  138       244 not.FB  33     62    178    No     is.PM    short     heavy
## 8  132       245 not.FB  23     65    140    No     is.PM     tall     heavy
## 9  120       289 not.FB  25     62    125    No    not.PM    short     light
## 10 143       299 not.FB  30     66    136   Yes    not.PM     tall     heavy
\end{verbatim}
\end{kframe}\begin{figure}[]


{\centering \includegraphics[width=.6\linewidth]{figure/latex-ass1-1} 

}

\caption[Initial check of the dataset]{Initial check of the dataset\label{fig:ass1-1}}
\end{figure}


\end{knitrout}


We can get an initial feel for the data by looking at the histogram of
one of the important variable, \textit{gestation}. The proportion of
premature pregnancy (shorter than 259 days) is around 30 percent, which is suitable for
analysis done on grouping the mothers based on this particular
variable. Similar initial checks are also done but will not be shown
here due to redundancy.


\newpage
\section*{Claim 1}
\subsection*{Mothers who smoke deliver premature babies more often
  than mothers who do not.}

\hspace{12 pt} Long story short, this claim is NOT well supported with
this dataset.

\subsubsection*{1. Graphical comparisons of the gestation distribution}
\hspace{12 pt} I tried two different ways of graphical comparisons,
histogram in parallel and group boxplot, both shown below. The gestation distributions
of smoking mothers and of non-smoking mothers DO NOT show a
significant difference in either mean or spread. 

\begin{knitrout}
\definecolor{shadecolor}{rgb}{0.969, 0.969, 0.969}\color{fgcolor}\begin{figure}[]


{\centering \includegraphics[width=.6\linewidth]{figure/latex-ass1-111} 

}

\caption[Graphical comparisons of the gestation distribution]{Graphical comparisons of the gestation distribution\label{fig:ass1-111}}
\end{figure}

\begin{figure}[]


{\centering \includegraphics[width=.6\linewidth]{figure/latex-ass1-112} 

}

\caption[Graphical comparisons of the gestation distribution]{Graphical comparisons of the gestation distribution\label{fig:ass1-112}}
\end{figure}


\end{knitrout}


\subsubsection*{2. Tabular comparisons of the factor variables}
\hspace{12 pt} With the added two-level factor variable,
\textit{premature}, indicating whether the baby was born prematurely,
we use the two factors, \textit{smoke} and \textit{premature}, to
carry out a relevant tabular comparison of distributions with results
shown below. 

\begin{knitrout}
\definecolor{shadecolor}{rgb}{0.969, 0.969, 0.969}\color{fgcolor}\begin{kframe}
\begin{alltt}
tb.PS <- \hlfunctioncall{table}(premature, smoke)
\hlfunctioncall{ftable}(tb.PS)
\end{alltt}
\begin{verbatim}
##           smoke  No Yes
## premature              
## not.PM          677 439
## is.PM            56  41
\end{verbatim}
\end{kframe}
\end{knitrout}


\subsubsection*{3. Visual tabular comparisons}
\hspace{12 pt} The figure shown below allows us to carry out visually the
tabular comparisons of the factor variables, \textit{smoke} and \textit{premature}.
We can NOT tell significant difference in the distribution of
gestation between smoking and non-smoking groups from the figure.
\begin{knitrout}
\definecolor{shadecolor}{rgb}{0.969, 0.969, 0.969}\color{fgcolor}\begin{figure}[]


{\centering \includegraphics[width=.6\linewidth]{figure/latex-ass1-13} 

}

\caption[Visual tabular comparisons]{Visual tabular comparisons\label{fig:ass1-13}}
\end{figure}


\end{knitrout}


\subsubsection*{4. Hypothesis test on the tabular comparison}
\hspace{12 pt} With the given table shown above, we want to conduct
hypothesis test formally for our claim. \newline


\hspace{12 pt}\textit{H0: smoking and non-smoking mothers have the
  same rate of premature delivery} \newline
\vspace{2 pt} 
\hspace{24 pt}\textit{HA: smoking and non-smoking mothers DO NOT have the
  same rate of premature delivery} \newline


Two tests are conducted, $Chi-squre$ and $Fisher$, and both have shown a
fairly large p-value, which indicates that we should not reject the
null hypothesis.

\begin{knitrout}
\definecolor{shadecolor}{rgb}{0.969, 0.969, 0.969}\color{fgcolor}\begin{kframe}
\begin{alltt}
\hlfunctioncall{chisq.test}(tb.PS)
\end{alltt}
\begin{verbatim}
## 
## 	Pearson's Chi-squared test with Yates' continuity correction
## 
## data:  tb.PS 
## X-squared = 0.2098, df = 1, p-value = 0.6469
\end{verbatim}
\newpage
\begin{alltt}
\hlfunctioncall{fisher.test}(tb.PS)
\end{alltt}
\begin{verbatim}
## 
## 	Fisher's Exact Test for Count Data
## 
## data:  tb.PS 
## p-value = 0.5893
## alternative hypothesis: true odds ratio is not equal to 1 
## 95 percent confidence interval:
##  0.7221 1.7530 
## sample estimates:
## odds ratio 
##      1.129
\end{verbatim}
\end{kframe}
\end{knitrout}


\subsubsection*{5. Hypothesis test on the overall average comparison}
\hspace{12 pt} With the data, we want to conduct
hypothesis test formally for a related question concerning the overall
average gestation time for the \textit{smoke} groups. \newline


\hspace{12 pt} \textit{H0: smoking and non-smoking mothers have the same overall average gestation time} \newline
\vspace{2 pt}
\hspace{24 pt} \textit{HA: smoking mothers have shorter overall average gestation time} \newline


Similarly, two one-sided tests are conducted, $t$ and $Wilcox$, and both have shown a
fairly large p-value, which indicates that we should not reject the
null hypothesis.

\begin{knitrout}
\definecolor{shadecolor}{rgb}{0.969, 0.969, 0.969}\color{fgcolor}\begin{kframe}
\begin{alltt}
\hlfunctioncall{t.test}(gestation ~ smoke, alternative = \hlstring{"less"})
\end{alltt}
\begin{verbatim}
## 
## 	Welch Two Sample t-test
## 
## data:  gestation by smoke 
## t = 2.394, df = 1093, p-value = 0.9916
## alternative hypothesis: true difference in means is less than 0 
## 95 percent confidence interval:
##   -Inf 3.726 
## sample estimates:
##  mean in group No mean in group Yes 
##             280.2             278.0
\end{verbatim}
\begin{alltt}
\hlfunctioncall{wilcox.test}(gestation ~ smoke, alternative = \hlstring{"less"})
\end{alltt}
\begin{verbatim}
## 
## 	Wilcoxon rank sum test with continuity correction
## 
## data:  gestation by smoke 
## W = 195867, p-value = 0.9996
## alternative hypothesis: true location shift is less than 0
\end{verbatim}
\end{kframe}
\end{knitrout}





\newpage
\section*{Claim 2}
\subsection*{Cigarette smoking has a stronger relationship to infant
  birth weight than serveral other relevant covariates.}

\hspace{12 pt} Long story short, this dataset DOES support this claim.

\subsubsection*{1. First-borns comparisons}
\hspace{12 pt} With the data, we want to conduct
hypothesis test formally for comparing the influence from relevant
variables on birth-weight \textit{bwt}. \newline


\hspace{12 pt} \textit{H0: the difference in the average bwt between smoking/non-smoking mothers is the same as that of firt-borns/non-first-borns} \newline
\vspace{2 pt}
\hspace{24 pt} \textit{HA: the difference in the average bwt between
  smoking/non-smoking mothers is NOT the same as that of firt-borns/non-first-borns} \newline


With the assumption of $i.i.d$ samples, we conduct a Wald test on the
groups, that is, \newline
$w = \displaystyle \frac{|\delta A| - |\delta B|}{se(\delta A - \delta  B)}$\newline
$\delta A=\bar X_{smoker} - \bar X_{non-smoker}; 
\delta B=\bar X_{first-born} - \bar X_{non-first-born}$\newline
$var(\delta A) = var(\bar X_{smoker})+var(\bar X_{non-smoker});
var(\delta B) = var(\bar X_{first-born})+var(\bar X_{non-first-born});$\newline
$var(\delta A - \delta B) = var(\delta A) + var(\delta B) -
2cov(\delta A, \delta B)$\newline
The details of the covariance calculations are given in the R code. So
from this, we can get the test statistic and p-value for the test. The
results are given below in the table for better comparison.

\subsubsection*{2. Mother height comparisons}
\hspace{12 pt} Similar to 1., we conduct the test and show the results
in the table below.

\subsubsection*{3. Mother weight comparisons}
\hspace{12 pt} Similar to 1., we conduct the test and show the results
in the table below.


\begin{table}
\caption{Comparison of univariate influence against smoke} \label{tab:title}
\begin{center}
\begin{tabular}{|l|l|l|}
Variable & Wald-test statistic & p-value (Normal) \\ \hline
{$parity$} & 0.2851 & 0.3878 \\ 
{$height$} & 0.1185 & 0.4528 \\ 
{$weight$} & 0.1460 & 0.4419 \\ 
\end{tabular}
\end{center}
\end{table}


\subsubsection*{4. Visual comparison on distributions}
\hspace{12 pt} In addition to the given table shown above, we want to
make multi-panel comparions of the whole distribution visually as the
figure shown below. 

\begin{knitrout}
\definecolor{shadecolor}{rgb}{0.969, 0.969, 0.969}\color{fgcolor}\begin{figure}[]


{\centering \includegraphics[width=.8\linewidth]{figure/latex-ass1-24} 

}

\caption[Difference  in bwt comparison of relevant variables]{Difference  in bwt comparison of relevant variables\label{fig:ass1-24}}
\end{figure}


\end{knitrout}


\subsubsection*{5. Multiple linear regression without smoking status}
\hspace{12 pt} With the data, we fit a linear regression model


\hspace{24 pt}
$bwt_1 = \beta_{1,0} + \beta_{1,1}height + \beta_{1,2}weight + \beta_{1,3}parity$. \newline
The summary is shown below and we check the fit by two plots. The one
with $fitted \sim residual$ is for checking the linear model $Y \sim
N(\beta^Tx,\sigma)$; while the histogram of the residuals from the
fitting is for checking the normality assumption of the residual
distribution. From visual inspection, both assumptions are satisfied.

\begin{knitrout}
\definecolor{shadecolor}{rgb}{0.969, 0.969, 0.969}\color{fgcolor}\begin{figure}[]


{\centering \includegraphics[width=\maxwidth]{figure/latex-ass2-25} 

}

\caption[Model 1 fitting check]{Model 1 fitting check\label{fig:ass2-25}}
\end{figure}


\end{knitrout}


\subsubsection*{6. Multiple linear regression with smoking status}
\hspace{12 pt} With the data, we fit another linear regression model


\hspace{24 pt}
$bwt_1 = \beta_{1,0} + \beta_{1,1}height + \beta_{1,2}weight +
\beta_{1,3}parity + \beta_{1,4}smoke$. \newline
The model is also checked for normality assumptions and is well satisfied.


We summarize the two models in the table below for easy informal
comparison. As shown in the table, the model 2 (with \textit{smoke} factor) has lower
residual standard error, higher $R^2$, lower p-value as compared to
the model 1 (without \textit{smoke} factor). We can conclude that M2
is better and that the \textit{smoke} factor DOES have a strong relationship
to infant birth weight \textit{bwt}.

\begin{table}
\caption{Comparison of linear regression models} \label{tab:title}
\begin{center}
\begin{tabular}{|l|l|l|}
Multiple Linear Regression & M1 : bwt~height+weight+parity & M2 : bwt~height+weight+parity+smoke \\ \hline
Residual standard error & {17.94 (df=1193)} & {17.34 (df=1182)} \\ 
{Multiple $R^2$} & 0.04695 & 0.1068 \\ 
{Adjusted $R^2$} & 0.04456 & 0.1038 \\ 
F-statistic & 19.59 & 35.33 \\ 
p-value & 2.112e-12 & <2.2e-16 \\
\end{tabular}
\end{center}
\end{table}


Meanwhile, ANOVA is carried out for formal comparison between the two
linear models. From the ANOVA results, we can see that \textit{smoke}
explains the majority sum of squres (SS) compared to all the other
relevant variables (\textit{height,weight,parity}), with the smallest
p-value. This also supports our claim 2.

\newpage
\begin{knitrout}
\definecolor{shadecolor}{rgb}{0.969, 0.969, 0.969}\color{fgcolor}\begin{kframe}
\begin{alltt}
\hlfunctioncall{anova}(fit1)
\end{alltt}
\begin{verbatim}
## Analysis of Variance Table
## 
## Response: bwt
##             Df Sum Sq Mean Sq F value  Pr(>F)    
## height       1  15888   15888    49.4 3.5e-12 ***
## weight       1   2348    2348     7.3   0.007 ** 
## parity       1    675     675     2.1   0.148    
## Residuals 1193 383827     322                    
## ---
## Signif. codes:  0 '***' 0.001 '**' 0.01 '*' 0.05 '.' 0.1 ' ' 1
\end{verbatim}
\begin{alltt}
\hlfunctioncall{anova}(fit2)
\end{alltt}
\begin{verbatim}
## Analysis of Variance Table
## 
## Response: bwt
##             Df Sum Sq Mean Sq F value  Pr(>F)    
## height       1  16110   16110   53.57 4.6e-13 ***
## weight       1   2278    2278    7.57   0.006 ** 
## parity       1    628     628    2.09   0.149    
## smoke        1  23478   23478   78.08 < 2e-16 ***
## Residuals 1182 355435     301                    
## ---
## Signif. codes:  0 '***' 0.001 '**' 0.01 '*' 0.05 '.' 0.1 ' ' 1
\end{verbatim}
\end{kframe}
\end{knitrout}



\subsubsection*{7. Pros and Cons of multiple-regression approach}
\hspace{12 pt} As compared to the univariate comparisons we carried
out initially, the multiple-regression approach gives more accurate
results of the relative influence on the birth-weight among serveral
relevant variable. Null acceptance can be derived purely from the p-values of the Wald
tests for univariate comparison, which seems to underestimate the
relationship of $smoke$ to $bwt$. The multiple regression models have
shown us that the $smoke$ is the single most influential variable for
the $bwt$ changes. However, the regression approach may be overly
optimistic about the claim as visual inspection on the plots does not
give that much confidence.

 


\end{document}
