\documentclass{article}

\usepackage{color}
\definecolor{dkgreen}{rgb}{0,0.6,0}
\definecolor{gray}{rgb}{0.5,0.5,0.5}
\definecolor{mauve}{rgb}{0.58,0,0.82}
\usepackage[margin=1in]{geometry}
\usepackage{fancyhdr}
\pagestyle{fancy}
\lhead{\today}
\chead{Ying Qiao SID:21412301}
\rhead{Stat215B Sp13: Assignment 1}
\lfoot{}
\cfoot{\thepage}
\rfoot{}
\usepackage{graphicx}
\usepackage{textcomp}
\usepackage{lmodern}
\usepackage[T1]{fontenc}
\usepackage{listings}

%% for inline R code: if the inline code is not correctly parsed, you will see a message
\newcommand{\rinline}[1]{SOMETHING WORNG WITH knitr}
%% begin.rcode setup, include=FALSE
% opts_chunk$set(fig.path='figure/latex-', cache.path='cache/latex-')
%% end.rcode


\begin{document}
\section*{Data Preparation}

\hspace{12 pt} Within \textit{babies.data}, there are 1236 observations of 7 
important variables related to the live births.
The data cleaning procedure is mainly focused on the missing value
removal and catagorical variable conversion. 
Among several variables, for example, \textit{gestation} indicating the days
of pregnancy, labels missing values with the number of 999. On the other hand,
\textit{parity} indicating whether the baby is the first-born, needs
to be coded categorically with description of levels. 
The initial cleaning results on the data frame \textit{babies} are
shown below.


%%  begin.rcode ass1-00, cache=TRUE, echo=FALSE, results="markup", warning=FALSE
%%babies <- read.table("babies.data", header=T)
%%babies$gestation[babies$gestation == 999] <- as.numeric("NA")
%%babies$age[babies$age == 99] <- as.numeric("NA")
%%babies$height[babies$height == 99] <- as.numeric("NA")
%%babies$weight[babies$weight == 999] <- as.numeric("NA")
%%babies$smoke[babies$smoke == 9] <- as.numeric("NA")
%%babies$parity <- factor(babies$parity)
%%levels(babies$parity) <- c("not.FB","is.FB") # first-born
%%babies$smoke <- factor(babies$smoke)
%%levels(babies$smoke) <- c("No","Yes") # smokes
%%head(babies, n = 10)
%%  end.rcode

When working on the statistical claims, we need to add serveral categorical
variables into the current data frame.
\textit{premature} is added as a two-level factor variable to indicate
whether a baby was born prematurely; and a premature birth is defined
as occuring prior to the 37th week ($37\times 7 = 259$ days) of
gestation. Similarly, \textit{f.height} and \textit{f.weight} are
added to divide the mothers into groups based on median height and
weight in the data. So the final clean data before any analysis is
sketched below.

%%  begin.rcode ass1-01, cache=TRUE, echo=FALSE,results="markup",warning=FALSE,dev='pdf',fig.align="center",out.width=".6\\linewidth",fig.cap="Initial check of the dataset"
%%babies$premature <- factor(babies$gestation < 37*7)
%%levels(babies$premature) <- c("not.PM","is.PM") # premature factor

%%babies$f.height <- factor(babies$height > median(babies$height, na.rm=T))
%%levels(babies$f.height) <- c("short","tall") # height factor

%%babies$f.weight <- factor(babies$weight > median(babies$weight, na.rm=T))
%%levels(babies$f.weight) <- c("light","heavy") # height factor

%%head(babies, n = 10)
%%hist(babies$gestation,col="blue",lwd=2, main="Distribution of overall gestation",xlab="Days of pregnancy",ylab="Counts of mothers")
%%  end.rcode

We can get an initial feel for the data by looking at the histogram of
one of the important variable, \textit{gestation}. The proportion of
premature pregnancy (shorter than 259 days) is around 30 percent, which is suitable for
analysis done on grouping the mothers based on this particular
variable. Similar initial checks are also done but will not be shown
here due to redundancy.


\newpage
\section*{Claim 1}
\subsection*{Mothers who smoke deliver premature babies more often
  than mothers who do not.}

\hspace{12 pt} Long story short, this claim is NOT well supported with
this dataset.

\subsubsection*{1. Graphical comparisons of the gestation distribution}
\hspace{12 pt} I tried two different ways of graphical comparisons,
histogram in parallel and group boxplot, both shown below. The gestation distributions
of smoking mothers and of non-smoking mothers DO NOT show a
significant difference in either mean or spread. 

%%  begin.rcode ass1-11, cache=TRUE, echo=FALSE,results="markup",message=FALSE,dev='pdf',fig.align="center",out.width=".6\\linewidth",fig.cap="Graphical comparisons of the gestation distribution"
%%attach(babies)
%%xlim=range(gestation, na.rm=T); ylim=c(0,0.04);
%%lwd=2;col="blue"; xlab="Days of pregnancy"; breaks=seq(140,360,10)
%%boxplot(gestation~smoke, data=babies, col=col,lwd=lwd, xlab="Smoke", ylab=xlab)
%%par(mfrow=c(2,1),cex.main= 1)
%%hist(gestation[smoke == "No"],breaks=breaks,freq=F, xlim=xlim,ylim=ylim, col=col,lwd=lwd,main="",xlab=xlab,ylab="Non-smoking mother")
%%hist(gestation[smoke == "Yes"],breaks=breaks,freq=F, xlim=xlim,ylim=ylim, col=col,lwd=lwd,main="",xlab=xlab,ylab="Smoking mother")
%%  end.rcode

\subsubsection*{2. Tabular comparisons of the factor variables}
\hspace{12 pt} With the added two-level factor variable,
\textit{premature}, indicating whether the baby was born prematurely,
we use the two factors, \textit{smoke} and \textit{premature}, to
carry out a relevant tabular comparison of distributions with results
shown below. 

%%  begin.rcode ass1-12, cache=TRUE, results="markup", message=FALSE, dev='pdf', fig.align="center",out.width=".6\\linewidth"
%%tb.PS <- table(premature, smoke)
%%ftable(tb.PS)
%%  end.rcode

\subsubsection*{3. Visual tabular comparisons}
\hspace{12 pt} The figure shown below allows us to carry out visually the
tabular comparisons of the factor variables, \textit{smoke} and \textit{premature}.
We can NOT tell significant difference in the distribution of
gestation between smoking and non-smoking groups from the figure.
%%  begin.rcode ass1-13, cache=TRUE, echo=FALSE, results="markup",message=FALSE, dev='pdf',fig.align="center",out.width=".6\\linewidth", fig.cap="Visual tabular comparisons"
%%require(stats)
%%mosaicplot(smoke ~ premature, data=babies, main = "Gestation distribution", color = TRUE)
%%  end.rcode

\subsubsection*{4. Hypothesis test on the tabular comparison}
\hspace{12 pt} With the given table shown above, we want to conduct
hypothesis test formally for our claim. \newline


\hspace{12 pt}\textit{H0: smoking and non-smoking mothers have the
  same rate of premature delivery} \newline
\vspace{2 pt} 
\hspace{24 pt}\textit{HA: smoking and non-smoking mothers DO NOT have the
  same rate of premature delivery} \newline


Two tests are conducted, $Chi-squre$ and $Fisher$, and both have shown a
fairly large p-value, which indicates that we should not reject the
null hypothesis.

%%  begin.rcode ass1-14, cache=TRUE, results="markup", message=FALSE, dev='pdf', fig.align="center",out.width=".6\\linewidth"
%%chisq.test(tb.PS)
%%fisher.test(tb.PS)
%%  end.rcode

\subsubsection*{5. Hypothesis test on the overall average comparison}
\hspace{12 pt} With the data, we want to conduct
hypothesis test formally for a related question concerning the overall
average gestation time for the \textit{smoke} groups. \newline


\hspace{12 pt} \textit{H0: smoking and non-smoking mothers have the same overall average gestation time} \newline
\vspace{2 pt}
\hspace{24 pt} \textit{HA: smoking mothers have shorter overall average gestation time} \newline


Similarly, two one-sided tests are conducted, $t$ and $Wilcox$, and both have shown a
fairly large p-value, which indicates that we should not reject the
null hypothesis.

%%  begin.rcode ass1-15, cache=TRUE, results="markup", message=FALSE, dev='pdf', fig.align="center",out.width=".6\\linewidth"
%%t.test(gestation ~ smoke, alternative = "less")
%%wilcox.test(gestation ~ smoke, alternative = "less")
%%  end.rcode




\newpage
\section*{Claim 2}
\subsection*{Cigarette smoking has a stronger relationship to infant
  birth weight than serveral other relevant covariates.}

\hspace{12 pt} Long story short, this dataset DOES support this claim.

\subsubsection*{1. First-borns comparisons}
\hspace{12 pt} With the data, we want to conduct
hypothesis test formally for comparing the influence from relevant
variables on birth-weight \textit{bwt}. \newline


\hspace{12 pt} \textit{H0: the difference in the average bwt between smoking/non-smoking mothers is the same as that of firt-borns/non-first-borns} \newline
\vspace{2 pt}
\hspace{24 pt} \textit{HA: the difference in the average bwt between
  smoking/non-smoking mothers is NOT the same as that of firt-borns/non-first-borns} \newline


With the assumption of $i.i.d$ samples, we conduct a Wald test on the
groups, that is, \newline
$w = \displaystyle \frac{|\delta A| - |\delta B|}{se(\delta A - \delta  B)}$\newline
$\delta A=\bar X_{smoker} - \bar X_{non-smoker}; 
\delta B=\bar X_{first-born} - \bar X_{non-first-born}$\newline
$var(\delta A) = var(\bar X_{smoker})+var(\bar X_{non-smoker});
var(\delta B) = var(\bar X_{first-born})+var(\bar X_{non-first-born});$\newline
$var(\delta A - \delta B) = var(\delta A) + var(\delta B) -
2cov(\delta A, \delta B)$\newline
The details of the covariance calculations are given in the R code. So
from this, we can get the test statistic and p-value for the test. The
results are given below in the table for better comparison.

\subsubsection*{2. Mother height comparisons}
\hspace{12 pt} Similar to 1., we conduct the test and show the results
in the table below.

\subsubsection*{3. Mother weight comparisons}
\hspace{12 pt} Similar to 1., we conduct the test and show the results
in the table below.


\begin{table}
\caption{Comparison of univariate influence against smoke} \label{tab:title}
\begin{center}
\begin{tabular}{|l|l|l|}
Variable & Wald-test statistic & p-value (Normal) \\ \hline
{$parity$} & 0.2851 & 0.3878 \\ 
{$height$} & 0.1185 & 0.4528 \\ 
{$weight$} & 0.1460 & 0.4419 \\ 
\end{tabular}
\end{center}
\end{table}


\subsubsection*{4. Visual comparison on distributions}
\hspace{12 pt} In addition to the given table shown above, we want to
make multi-panel comparions of the whole distribution visually as the
figure shown below. 

%%  begin.rcode ass1-24, cache=TRUE, echo=FALSE, results="markup",message=FALSE, dev='pdf',fig.align="center",out.width=".8\\linewidth", fig.cap="Difference  in bwt comparison of relevant variables"
%%par(mfrow=c(2,2),cex.main= 1)
%%ylab="Birth weight";ylim=range(bwt,na.rm=T)
%%boxplot(bwt~smoke, data=babies, ylim=ylim, col=col,lwd=lwd, xlab="Smoke", ylab=ylab)
%%boxplot(bwt~parity, data=babies, ylim=ylim, col=col,lwd=lwd, xlab="First-born", ylab=ylab)
%%boxplot(bwt~f.height, data=babies, ylim=ylim, col=col,lwd=lwd, xlab="Height", ylab=ylab)
%%boxplot(bwt~f.weight, data=babies, ylim=ylim, col=col,lwd=lwd, xlab="Weight", ylab=ylab)
%%  end.rcode

\subsubsection*{5. Multiple linear regression without smoking status}
\hspace{12 pt} With the data, we fit a linear regression model


\hspace{24 pt}
$bwt_1 = \beta_{1,0} + \beta_{1,1}height + \beta_{1,2}weight + \beta_{1,3}parity$. \newline
The summary is shown below and we check the fit by two plots. The one
with $fitted \sim residual$ is for checking the linear model $Y \sim
N(\beta^Tx,\sigma)$; while the histogram of the residuals from the
fitting is for checking the normality assumption of the residual
distribution. From visual inspection, both assumptions are satisfied.

%%  begin.rcode ass2-25, cache=TRUE, echo=FALSE, results="markup",message=FALSE, dev='pdf',fig.align="center",fig.width=6,fig.height=4,fig.cap="Model 1 fitting check"
%%fit1 = lm(bwt ~ height + weight + parity, na.action= na.exclude)
%%par(mfrow=c(1,2),cex.main= 1)
%%plot(fitted(fit1), resid(fit1), type="p", col=col, lwd=lwd, xlab="Fitted values of Model 1", ylab="Residuals of Model 1", main="Fitting check")
%%lim=max(resid(fit1), na.rm=T)
%%hist(resid(fit1),breaks=20,freq=F,xlim=c(-lim,lim),ylim=c(0,0.025), col=col,lwd=lwd,main="Normality check",xlab="Residual values of Model 1",ylab="Density")
%%  end.rcode

\subsubsection*{6. Multiple linear regression with smoking status}
\hspace{12 pt} With the data, we fit another linear regression model


\hspace{24 pt}
$bwt_1 = \beta_{1,0} + \beta_{1,1}height + \beta_{1,2}weight +
\beta_{1,3}parity + \beta_{1,4}smoke$. \newline
The model is also checked for normality assumptions and is well satisfied.


We summarize the two models in the table below for easy informal
comparison. As shown in the table, the model 2 (with \textit{smoke} factor) has lower
residual standard error, higher $R^2$, lower p-value as compared to
the model 1 (without \textit{smoke} factor). We can conclude that M2
is better and that the \textit{smoke} factor DOES have a strong relationship
to infant birth weight \textit{bwt}.

\begin{table}
\caption{Comparison of linear regression models} \label{tab:title}
\begin{center}
\begin{tabular}{|l|l|l|}
Multiple Linear Regression & M1 : bwt~height+weight+parity & M2 : bwt~height+weight+parity+smoke \\ \hline
Residual standard error & {17.94 (df=1193)} & {17.34 (df=1182)} \\ 
{Multiple $R^2$} & 0.04695 & 0.1068 \\ 
{Adjusted $R^2$} & 0.04456 & 0.1038 \\ 
F-statistic & 19.59 & 35.33 \\ 
p-value & 2.112e-12 & <2.2e-16 \\
\end{tabular}
\end{center}
\end{table}


Meanwhile, ANOVA is carried out for formal comparison between the two
linear models. From the ANOVA results, we can see that \textit{smoke}
explains the majority sum of squres (SS) compared to all the other
relevant variables (\textit{height,weight,parity}), with the smallest
p-value. This also supports our claim 2.
%%  begin.rcode ass2-26, cache=TRUE, echo=FALSE, results="markup",message=FALSE, dev='pdf',fig.align="center"
%%fit2 = lm(bwt ~ height + weight + parity + smoke, na.action= na.exclude)
%%  end.rcode
%%  begin.rcode ass2-26-a, cache=TRUE, results="markup",message=FALSE, dev='pdf',fig.align="center"
%%anova(fit1)
%%anova(fit2)
%%  end.rcode


\subsubsection*{7. Pros and Cons of multiple-regression approach}
\hspace{12 pt} As compared to the univariate comparisons we carried
out initially, the multiple-regression approach gives more accurate
results of the relative influence on the birth-weight among serveral
relevant variable. Null acceptance can be derived purely from the p-values of the Wald
tests for univariate comparison, which seems to underestimate the
relationship of $smoke$ to $bwt$. The multiple regression models have
shown us that the $smoke$ is the single most influential variable for
the $bwt$ changes. However, the regression approach may be overly
optimistic about the claim as visual inspection on the plots does not
give that much confidence.

 


\end{document}
