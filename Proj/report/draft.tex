\documentclass[12pt]{article}
\usepackage{latexsym,amssymb,amsmath} % for \Box, \mathbb, split, etc.
\usepackage{bm} % for bold math
% \usepackage[]{showkeys} % shows label names
\usepackage{cite} % sorts citation numbers appropriately
\usepackage{path}
\usepackage{url}
\usepackage{verbatim}
\usepackage[pdftex]{graphicx}

% horizontal margins: 1.0 + 6.5 + 1.0 = 8.5
\setlength{\oddsidemargin}{0.0in}
\setlength{\textwidth}{6.5in}
% vertical margins: 1.0 + 9.0 + 1.0 = 11.0
\setlength{\topmargin}{0.0in}
\setlength{\headheight}{12pt}
\setlength{\headsep}{13pt}
\setlength{\textheight}{625pt}
\setlength{\footskip}{24pt}

\renewcommand{\textfraction}{0.10}
\renewcommand{\topfraction}{0.85}
\renewcommand{\bottomfraction}{0.85}
\renewcommand{\floatpagefraction}{0.90}

\makeatletter
\setlength{\arraycolsep}{2\p@} % make spaces around "=" in eqnarray smaller
\makeatother

% change equation, table, figure numbers to be counted inside a section:
\numberwithin{equation}{section}
\numberwithin{table}{section}
\numberwithin{figure}{section}

% begin of personal macros
\newcommand{\half}{{\textstyle \frac{1}{2}}}
\newcommand{\eps}{\varepsilon}
\newcommand{\myth}{\vartheta}
\newcommand{\myphi}{\varphi}

\newcommand{\IN}{\mathbb{N}}
\newcommand{\IZ}{\mathbb{Z}}
\newcommand{\IQ}{\mathbb{Q}}
\newcommand{\IR}{\mathbb{R}}
\newcommand{\IC}{\mathbb{C}}
\newcommand{\Real}[1]{\mathrm{Re}\left({#1}\right)}
\newcommand{\Imag}[1]{\mathrm{Im}\left({#1}\right)}

\newcommand{\norm}[2]{\|{#1}\|_{{}_{#2}}}
\newcommand{\abs}[1]{\left|{#1}\right|}
\newcommand{\ip}[2]{\left\langle {#1}, {#2} \right\rangle}
\newcommand{\der}[2]{\frac{\partial {#1}}{\partial {#2}}}
\newcommand{\dder}[2]{\frac{\partial^2 {#1}}{\partial {#2}^2}}

\newcommand{\nn}{\mathbf{n}}
\newcommand{\xx}{\mathbf{x}}
\newcommand{\uu}{\mathbf{u}}

\newcommand{\junk}[1]{{}}

\newcommand{\bms}{\bm{s}}
\newcommand{\bmh}{\bm{h}}

% set two lengths for the includegraphics commands used to import the plots:
\newlength{\fwtwo} \setlength{\fwtwo}{0.45\textwidth}
% end of personal macros

\begin{document}
\DeclareGraphicsExtensions{.jpg}

\begin{center}
\textbf{\Large Enhanced Process Monitoring with Metrology Data} \\[6pt]
  Haotian Liu, Ying Qiao and Zhongwei Zhu \\[6pt]
  \{liuht05, yingqiao, zhongweizhu\}@berkeley.edu
\end{center}

\begin{abstract}
to be filled later ......
\end{abstract}

\subparagraph{Key words.} to be filled later ......


\section{Introduction}

\hspace{12 pt}
High-quality statistical process control is always important in
semiconductor processing in many high-volume manufacturing
industries\cite{DAC09}. Variability, which can arise from all steps in
semiconductor fabrication hierarchy, as shown in Figure \ref{metro_hier},
causes performance degradation 
in electrical circuits on the chips. Monitoring and modeling the
fabrication process is accordingly crucial for choosing the reliable 
chips at a low cost. Systematic spatial variation can cause spatial
correlation between structures on a die, for example, devices that are
in close proximity behave much more similarly than those spaced
farther apart\cite{KedarPHD}. Variogram-based modeling of spatial
correlation has been advocated in the recent literatures to
characterize the wafer-level spatial variability pattern. 


In this project, we aim at presenting a statistical framework for 
process monitoring based on an electrical metrology dataset from a
high-volume fabrication line. The proposed framework focuses on 
detecting wafers with disrupted spatial patterns, that is,
\emph{fault} wafers that contain a large portion of dies with distinct
electrical metrology results from other wafers. Early detection of
these spatially disrupted wafers during the manufacturing process can
greatly reduce the metrology cost and improve the overall chip performance yield.
Our FDC (Fault wafer Detection and Characterization) scheme is developed on the
``complete'' wafer dataset with missing die values interpolated using Kriging
method\cite{Cressie93}. Kernel PCA [cite] and one-class SVM [cite] method have
been applied to detect the fault wafers.   


The rest of the report is organized as follows: section \ref{kriging}
will offer a brief dicussion of spatial statistics and Kriging method
for missing data interpolation; section \ref{kernal} dicusses the PCA
(principle component analysis) and kernel PCA methods for FDC; section
\ref{svm} shows the one-class SVM procedure for FDC; lastly, dection
performance comparisons and concluding remarks are presented in
section \ref{result} and section \ref{summary}.

\begin{figure} \centering
  \includegraphics[width=0.6\textwidth]{metro_hier}
  \caption{A typical metrology hierarchy of semiconductor manufacturing}
  \label{metro_hier}
\end{figure}


\section{Missing Data Interpolation with Kriging} \label{kriging}

\hspace{12 pt}
In this section, we first offer a brief discussion of how spatial
statistics are used to characterize the spatial variation of wafers
and its application to interpolate the missing die values. Then we
explore the data structure of our electrical metrology
measurements and demonstrate the wafer dataset after missing data
interpolation. This ``complete'' dataset will be used in the following
FDC development.  


\subsection{Spatial Statistics for Kriging} \label{spatial}

\hspace{12 pt}
The spatial variation of circuit performance is introduced by the
nature of the fabrication process. For example, during plasma etching
operations, the center peak shape of RF (radio-frequency) electric
field distribution leads to a similar center peak shape of etch rate,
that is, etch rate varies radially across the wafer\cite{dkPHD}. Given
the assumption that electrical metrology data is intrinsically
\emph{stationary}, we can characterize each wafer using a spatial
variogram.


Let $\mathcal{X}_s = \{ X(\bms) : \bms \in \mathcal{R} \}$ be a
collection of random variables from a stationary spatial process in
region $\mathcal{R}$, then we have
\begin{equation} \label{ssmean}
  E[X(\bms) - X(\bms + \bmh)] = 0
\end{equation}
and
\begin{equation} \label{ssvar}
\begin{split}
  Var[X(\bms) - X(\bms + \bmh)] &= Var[X(\bm{0}) - X(\bmh)] \\
  &= E[X(\bm{0}) - X(\bmh)]^2 \\
  &= 2\gamma(\bmh)
\end{split}
\end{equation}
for all $\bms \in \mathcal{R}$. In spatial statistics, $2\gamma$ is
known as the \emph{variogram}. Here, for a stationary spatial process,
we can see that the covariance between the values at any two locations 
depends only on the separation vector $\bmh$. Defining a common
auto-covariance function $C(\bmh) = Cov(X(\bms), X(\bms +
\bmh))$, we can then express the variogram as
\begin{equation} \label{variodef}
\begin{split}
  2\gamma(\bmh) &= Var[X(\bms) - X(\bms + \bmh)] \\
  &= Var(X(\bm{0})) + Var(X(\bmh)) - 2Cov(X(\bms), X(\bms + \bmh)) \\
  &= 2[C(\bm{0}) - C(\bmh)]
\end{split}
\end{equation}


A version of empirical, or classical, estimation of variogram is given as
\begin{equation} \label{varioest0}
  2\hat{\gamma}(h) = \frac{1}{|N(h)|} \sum_{N(h)} \left( X(\bms_i)
  - X(\bms_j) \right)^2 
\end{equation}
where $h$ degrades to a scaler ignoring the directional information and 
\begin{equation} \label{Nhdef}
  N(h) \doteq \{ (\bms_i, \bms_j) : \norm{\bms_i - \bms_j}{2} = h; i,j = 1,2,...,n \}
\end{equation}
denotes the set of pairs of sample points with $h$ distance between
them, and $|N(h)|$ is the number of pairs in this set. Note that in
this classical estimator, the square operation inside the summation
greatly magnifies any outlier obervation, so a more robust estimator
was introduced by Cressie and Hawkins\cite{Cressie93}
\begin{equation} \label{varioest1}
  2\hat{\gamma}(h) = \frac{1}{0.457+\frac{0.494}{|N(h)||}}\left[\frac{1}{|N(h)|}
  \sum_{N(h)} (X(\bms_i) - X(\bms_j))^{1/2}\right]^4
\end{equation}


\begin{figure} \centering
  \includegraphics[width=0.6\textwidth]{vario_model}
  \caption{Illustration of parameters in the general semivariogram model}
  \label{vario_model}
\end{figure}


The variogram or semivariogram, $\gamma(h)$, 
is perferred here because it tends to filter the
influence of a spatially varying mean and thus avoids the estimation
of the mean. The estimated empirical variogram may not be positive
definite and hence not directly usable as weights in \emph{kriging}, 
without constraints or further processing. This explains why
only a limited number of valid parametric variogram models can be used.
A general form of variogram model for isotropic process is given by
\begin{equation} \label{variomod}
  \hat{\gamma}(h) = c_0 + (\sigma^2 - c_0)[ 1- \rho(h)]
\end{equation}
where the parametric auto-correlation function(ACF) can be expressed as 
\begin{equation} \label{acf}
  \rho(h) = \exp{ \left[-\left( \frac{h}{\xi} \right)^{2\phi} \right] }
\end{equation}
Figure \ref{vario_model} illustrates various parameters used to
describe the (semi)variogram. The \emph{nugget}, $c_0$, is an 
independent component of variation that does not depend on the lag
and accounts for the discontinuity at zero lag. The partial
\emph{sill}, $\sigma^2$, limits the variogram tending to infinite
lag. Also, the lag value at which variogram reaches the sill is
known as the \emph{range} (denoted as $r$ here). When applied to our
real dataset, we adopted a simpler cubic approximation 
of the ACF model in (\ref{acf}), and that gives the parametric variogram
model that we want to estimate as \cite{geoR},
\begin{equation} \label{cubic}
  2\hat{\gamma}(h;\bm{\theta}) = 
\begin{cases} 
  2c_0 + 2(\sigma^2 - c_0) 
  \left[ 7\left( \frac{h}{r} \right)^2 - 8.75\left( \frac{h}{r} \right)^3
  + 3.5\left( \frac{h}{r}\right)^5 - 0.75\left( \frac{h}{r}\right)^7
  \right] & \mbox{if $h<r$} \\
  2\sigma^2 & \mbox{otherwise}
\end{cases}
\end{equation}
Let $\bm{\theta} \doteq (c_0, \sigma^2, r)$ denotes the vector of
parameters that need to be estimated for characterizing the
variogram. In this work, we will employ the weighted least square method
to get the estimator,
\begin{equation} \label{variowls}
  \hat{\bm{\theta}} = arg\min_{\theta} \sum_h |N(h)|\left[
    2\hat{\gamma}(h; \bm{\theta}) - 2\hat{\gamma}(h) \right]^2
\end{equation}
where $N(h)$ is given by (\ref{Nhdef}), $2\hat{\gamma}(h)$ and
$2\hat{\gamma}(h;\bm{\theta})$ are given by (\ref{varioest1}) and
(\ref{cubic}) respectively. We can thereafter use this model in
ordinary kriging method for missing data interpolation. 


Interpolating missing values by kriging is based on the idea that the
value at an unknown point should be a weighted average of the known
values at its neighbors \cite{Cressie93}. 
The spatial inference of $X$ at an unobserved location $\bms_0$ is
calculated from a linear combination of the observed neighboring
values $\{X(\bms_i), i=1,2,...,n\}$,
\begin{equation} \label{nawls}
  \hat{X}(\bms_0) = \sum_{i=1}^n w_i(\bms_0)X(\bms_i) = \bm{W}^T\bm{X}_n
\end{equation}
where $\bm{W} = \left[ w_1(\bms_0) \quad w_2(\bms_0) \quad \cdots 
\quad w_n(\bms_0) \right]^T$, and 
$\bm{X}_n = \left[ X(\bms_1) \quad X(\bms_2) \quad \cdots
\quad X(\bms_n) \right]^T$. The error term is therefore declared as
\begin{equation} \label{errordef}
  \epsilon(\bms_0) \doteq \hat{X}(\bms_0) - X(\bms_0) 
  = \left[ \bm{W}^T \quad -1 \right] \cdot \left[ \bm{X}_n \quad X(\bms_0) \right]^T
\end{equation}
To get a best linear unbiased estimator, the kriging linear system
will be the result of this optimization problem,
\begin{equation} \label{krigsys}
\begin{split}
  \min_{\bm{W}} \quad & Var(\epsilon(\bms_0))\\
  \mbox{s.t.} \quad & E(\epsilon(\bms_0)) = 0
\end{split}
\end{equation}


In \emph{ordinary kriging}, stationarity of the first moment of
all random variables is assumed, $ E[X(\bms_i)] = E[X(\bms_0)] = m$,
where $m$ is unknown. So the constraint of unbiasness is 
\begin{equation} \label{unbias}
\begin{split}
  E(\epsilon(\bms_0)) = 0 \quad 
  \Leftrightarrow \quad &E(\bm{W}^T\bm{X}_n - X(\bms_0)) = 0 \\
  \Leftrightarrow \quad &\bm{W}^TE(\bm{X}_n) = E(X(\bms_0)) \\
  \Leftrightarrow \quad &\bm{W}^T\bm{1} = 1
\end{split}
\end{equation}
and the error variance is
\begin{equation} \label{varerr}
\begin{split}
  Var(\epsilon(\bms_0)) &= Var(\hat{X}(\bms_0) - X(\bms_0)) \\
  &=[\bm{W}^T \quad -1] \cdot 
  \begin{bmatrix}
    Var(\bm{X}_n^T) &  Cov(\bm{X}_n^T,  X(\bms_0))\\
    \\
    Cov^T(\bm{X}_n^T,  X(\bms_0)) & Var(X(\bms_0))
  \end{bmatrix}
  \cdot \begin{bmatrix}
    \bm{W} \\
    \\
    -1
  \end{bmatrix} \\
 &=\bm{W}^T\mathbb{V}(\bm{X}_n^T)\bm{W} - \mathbb{C}^T(\bm{X}_n^T, X(\bms_0))\bm{W}
  - \bm{W}^T\mathbb{C}(\bm{X}_n^T, X(\bms_0)) + \mathbb{V}(X(\bms_0))
\end{split}
\end{equation}
Solving the optimization problem results in the kriging system that
gives us $\hat{\bm{W}}$ and the estimate $\hat{X(\bms_0)}=\hat{\bm{W}}^T\bm{X}_n$.


\subsection{Missing Data Interpolation} \label{nafit}

\hspace{12 pt}
The data in this project comes from a high-volume manufacturing line
of 65 nm technology process at IBM. It consists of measurements on 348
wafers that span 23 lots, with an uneven number of wafers within each
lot. Each wafer contains 117 dies (also known as chips), with varied
numbers of dies not measured on each wafer, that is, missing die
values. 14 frequency measurements of ring oscillators were performed
at different locations within the die. Owing to the proprietary nature
of the data, all measured values are presented in arbitrary units
without loss in the generality of our proposed procedures. 
Figure \ref{wafer_counts} displays the wafer allocations among the
lots for simple visualization of the dataset. 


In our context, $X$ represents the mean of the 14 PSRO measurements
and $\bmh = \{ (x_i, y_i) \}$ is the set of chip locations on the wafer.
Figure \ref{vario_fit} shows a variogram model fitted to the
emprical variogram of a typical wafer (Lot 1 Wafer 7). Figure
\ref{wafer_fill} demonstrate the missing value interpolation performed
on that same wafer.

\begin{figure} \centering
  \includegraphics[width=0.4\textwidth]{wafer_counts}
  \tiny \caption{Wafer allocations for lot}
  \label{wafer_counts}
\end{figure}

\begin{figure} \centering
  \includegraphics[width=0.6\textwidth]{vario_fit}
  \tiny \caption{Emprical variogram(dot) and fitted parametric 
  variogram model(line) for a typical wafer (Lot 1 Wafer 7)}
  \label{vario_fit}
\end{figure}

\begin{figure} \centering
  \begin{tabular}{cc}
    \includegraphics[width=\fwtwo]{wafer_na} &
    \includegraphics[width=\fwtwo]{wafer_intpol} \\
    (a) & (b)
  \end{tabular}
  \caption{Plot of missing die value data interpolation,
  (a) raw data wafer metrology map, (b) after missing data
  interpolation using kriging}
  \label{wafer_fill}
\end{figure}













\section{Conclusions}

to be filled later ... 


\section*{Acknowledgments}

to be filled later ... 

 
\appendix

\section{Code for Plots} \label{appcode}


\verbatiminput{plot_loglog.m}

\bibliographystyle{siam}
\bibliography{draft}

\end{document}

