\documentclass{article}

\usepackage{color}
\definecolor{dkgreen}{rgb}{0,0.6,0}
\definecolor{gray}{rgb}{0.5,0.5,0.5}
\definecolor{mauve}{rgb}{0.58,0,0.82}
\usepackage[margin=1in]{geometry}
\usepackage{fancyhdr}
\pagestyle{fancy}
\lhead{\today}
\chead{Ying Qiao SID:21412301}
\rhead{Stat215B Sp13: Assignment 7}
\lfoot{}
\cfoot{\thepage}
\rfoot{}
\usepackage{graphicx}
\usepackage{textcomp}
\usepackage{lmodern}
\usepackage[T1]{fontenc}
\usepackage{listings}
\usepackage{amssymb,amsmath}
\usepackage{bm}
\usepackage{hhline}

%% for inline R code: if the inline code is not correctly parsed, you will see a message
\newcommand{\rinline}[1]{SOMETHING WORNG WITH knitr}
%% begin.rcode setup, include=FALSE
% opts_chunk$set(fig.path='figure/latex-', cache.path='cache/latex-')
%% end.rcode


\begin{document}
\section*{Math stats}
Efron (2010) exercises

\subsubsection*{2.5}
Within Lemma 2.1, we have two expectations of functions
\begin{displaymath}
\begin{split}
E\{\overline{Fdr}(\mathcal{Z}) | N_1(\mathcal{Z})\} 
& = E\{ \frac{e_0(\mathcal{Z})}{N_0(\mathcal{Z}) + N_1(\mathcal{Z})} |N_1(\mathcal{Z})\}
= E\{ \eta(N_0(\mathcal{Z})) | N_1(\mathcal{Z}) \} \\
E\{Fdp(\mathcal{Z}) | N_1(\mathcal{Z})\} 
& = E\{ \frac{N_0(\mathcal{Z})}{N_0(\mathcal{Z}) + N_1(\mathcal{Z})} |N_1(\mathcal{Z})\}
= E\{ \zeta(N_0(\mathcal{Z})) | N_1(\mathcal{Z}) \}
\end{split}
\end{displaymath}
where $\eta(N_0(\mathcal{Z}))$ is convex and $\zeta(N_0(\mathcal{Z}))$
is concave.
\subsubsection*{}
Meanwhile, we have one function of expectations
\begin{displaymath}
\phi_1(\mathcal{Z}) = \frac{e_0(\mathcal{Z})}{e_0(\mathcal{Z}) + N_1(\mathcal{Z})}
\end{displaymath}
here, with null case independence,
\begin{displaymath}
e_0(\mathcal{Z}) = E\{ N_0(\mathcal{Z}) \} = E\{ N_0(\mathcal{Z}) | N_1(\mathcal{Z})\}
\end{displaymath}
Then, we have
\begin{displaymath}
\begin{split}
\phi_1(\mathcal{Z}) &= \frac{e_0(\mathcal{Z})}{E\{N_0(\mathcal{Z})\} +  N_1(\mathcal{Z})} 
= \eta(E\{N_0(\mathcal{Z}) | N_1(\mathcal{Z})\}) \\
\phi_1(\mathcal{Z}) &= \frac{E\{N_0(\mathcal{Z})\}}{E\{N_0(\mathcal{Z})\} +  N_1(\mathcal{Z})} 
= \zeta(E\{N_0(\mathcal{Z}) | N_1(\mathcal{Z})\})
\end{split}
\end{displaymath}
Finally, with different function conditions for Jensen's inequality,
we have:\newline
for convex $\eta(\cdot)$,
\begin{displaymath}
\begin{split}
\eta(E\{N_0(\mathcal{Z}) | N_1(\mathcal{Z})\}) &\leq E\{\eta(N_0(\mathcal{Z})) | N_1(\mathcal{Z}) \} \\
\leftrightarrow
\phi_1(\mathcal{Z}) &\leq E\{\overline{Fdr}(\mathcal{Z}) | N_1(\mathcal{Z})\} 
\end{split}
\end{displaymath}
for concave $\zeta(\cdot)$,
\begin{displaymath}
\begin{split}
\zeta(E\{N_0(\mathcal{Z}) | N_1(\mathcal{Z})\}) &\geq E\{\zeta(N_0(\mathcal{Z})) | N_1(\mathcal{Z}) \} \\
\leftrightarrow
\phi_1(\mathcal{Z}) &\geq E\{Fdp(\mathcal{Z}) | N_1(\mathcal{Z})\} 
\end{split}
\end{displaymath}
That completes our proof for (2.30).







\newpage
\section*{Prostate microarray data}
\hspace{12 pt} We have the prostate microarray data from the Efron
book, a $6033 \times 102$ matrix $X=\{x_{ij}\}$.
\begin{center}
$x_{ij}$ = expression of gene $i$ on patient $j$, 
\end{center}
where $i=1,..,N$, $N=6033$; normal patients $j=1,...,50$ vs. cancer
patients $j=51,...,102$.
%%  begin.rcode ass6-00,cache=TRUE,results="markup",message=FALSE,echo=FALSE
%%rm(list=ls())
%%load("prostatedata.Rda")
%%N <- dim(prostatedata)[1]
%%  end.rcode


\subsubsection*{Efron Figure 2.1}
\hspace{12 pt} The large-scale hypothesis testing that we perform on
this dataset use the two-sample $t$-statistic. \newline
For testing gene $i$,
\begin{displaymath}
t_i = \frac{\bar{x}_i(2) - \bar{x}_i(1)}{s_i}
\end{displaymath}
\begin{displaymath}
\bar{x}_i(1) = \frac{1}{50} \sum_{j=1}^{50} x_{ij} ; \hspace{8 pt}
\bar{x}_i(2) = \frac{1}{52} \sum_{j=51}^{102} x_{ij}
\end{displaymath}
\begin{displaymath}
s_i^2 = \frac{\sum_{j=1}^{50} (x_{ij} - \bar{x}_i(1))^2 +
  \sum_{j=51}^{102} (x_{ij} - \bar{x}_i(2))^2 }{100} \cdot (\frac{1}{50} + \frac{1}{52}) 
\end{displaymath}
Then, we transform to the $z$-values, 
where $\Phi$ is the standard normal CDF and $F_{100}$ is the Student-$t$ CDF
with 100 degrees of freedom,
\begin{displaymath}
z_i = \Phi^{-1}(F_{100}(t_i))
\end{displaymath}
Finally, we reproduce Efron
Figure 2.1, histogram of $z$-values testing $N=6033$ genes
for possible involvement with prostate cancer.

%%  begin.rcode ass6-21,cache=TRUE,echo=FALSE,results="markup",dev='pdf',fig.height=6,fig.width=7,fig.align="center",fig.cap="Reproduced Efron Figure 2.1"
%%x1 <- rowMeans(prostatedata[ ,1:50])
%%x2 <- rowMeans(prostatedata[ ,51:102])
%%s2 <- (1/50 + 1/52)/100 * (rowSums((prostatedata[ ,1:50] - x1)^2) + rowSums((prostatedata[ ,51:102] - x2)^2))
%%t <- (x2 - x1) / sqrt(s2)
%%z <- qnorm(pt(t, 100)) #greater
%%hist(z, breaks=100, xlim=c(-5,5), xlab="z values->", ylab="Frequency", main="", col="grey")
%%  end.rcode



\end{document}
