\documentclass{article}\usepackage{graphicx, color}
%% maxwidth is the original width if it is less than linewidth
%% otherwise use linewidth (to make sure the graphics do not exceed the margin)
\makeatletter
\def\maxwidth{ %
  \ifdim\Gin@nat@width>\linewidth
    \linewidth
  \else
    \Gin@nat@width
  \fi
}
\makeatother

\IfFileExists{upquote.sty}{\usepackage{upquote}}{}
\definecolor{fgcolor}{rgb}{0.2, 0.2, 0.2}
\newcommand{\hlnumber}[1]{\textcolor[rgb]{0,0,0}{#1}}%
\newcommand{\hlfunctioncall}[1]{\textcolor[rgb]{0.501960784313725,0,0.329411764705882}{\textbf{#1}}}%
\newcommand{\hlstring}[1]{\textcolor[rgb]{0.6,0.6,1}{#1}}%
\newcommand{\hlkeyword}[1]{\textcolor[rgb]{0,0,0}{\textbf{#1}}}%
\newcommand{\hlargument}[1]{\textcolor[rgb]{0.690196078431373,0.250980392156863,0.0196078431372549}{#1}}%
\newcommand{\hlcomment}[1]{\textcolor[rgb]{0.180392156862745,0.6,0.341176470588235}{#1}}%
\newcommand{\hlroxygencomment}[1]{\textcolor[rgb]{0.43921568627451,0.47843137254902,0.701960784313725}{#1}}%
\newcommand{\hlformalargs}[1]{\textcolor[rgb]{0.690196078431373,0.250980392156863,0.0196078431372549}{#1}}%
\newcommand{\hleqformalargs}[1]{\textcolor[rgb]{0.690196078431373,0.250980392156863,0.0196078431372549}{#1}}%
\newcommand{\hlassignement}[1]{\textcolor[rgb]{0,0,0}{\textbf{#1}}}%
\newcommand{\hlpackage}[1]{\textcolor[rgb]{0.588235294117647,0.709803921568627,0.145098039215686}{#1}}%
\newcommand{\hlslot}[1]{\textit{#1}}%
\newcommand{\hlsymbol}[1]{\textcolor[rgb]{0,0,0}{#1}}%
\newcommand{\hlprompt}[1]{\textcolor[rgb]{0.2,0.2,0.2}{#1}}%

\usepackage{framed}
\makeatletter
\newenvironment{kframe}{%
 \def\at@end@of@kframe{}%
 \ifinner\ifhmode%
  \def\at@end@of@kframe{\end{minipage}}%
  \begin{minipage}{\columnwidth}%
 \fi\fi%
 \def\FrameCommand##1{\hskip\@totalleftmargin \hskip-\fboxsep
 \colorbox{shadecolor}{##1}\hskip-\fboxsep
     % There is no \\@totalrightmargin, so:
     \hskip-\linewidth \hskip-\@totalleftmargin \hskip\columnwidth}%
 \MakeFramed {\advance\hsize-\width
   \@totalleftmargin\z@ \linewidth\hsize
   \@setminipage}}%
 {\par\unskip\endMakeFramed%
 \at@end@of@kframe}
\makeatother

\definecolor{shadecolor}{rgb}{.97, .97, .97}
\definecolor{messagecolor}{rgb}{0, 0, 0}
\definecolor{warningcolor}{rgb}{1, 0, 1}
\definecolor{errorcolor}{rgb}{1, 0, 0}
\newenvironment{knitrout}{}{} % an empty environment to be redefined in TeX

\usepackage{alltt}

\usepackage{color}
\definecolor{dkgreen}{rgb}{0,0.6,0}
\definecolor{gray}{rgb}{0.5,0.5,0.5}
\definecolor{mauve}{rgb}{0.58,0,0.82}
\usepackage[margin=1in]{geometry}
\usepackage{fancyhdr}
\pagestyle{fancy}
\lhead{\today}
\chead{Ying Qiao SID:21412301}
\rhead{Stat215B Sp13: Assignment 3}
\lfoot{}
\cfoot{\thepage}
\rfoot{}
\usepackage{graphicx}
\usepackage{textcomp}
\usepackage{lmodern}
\usepackage[T1]{fontenc}
\usepackage{listings}

%% for inline R code: if the inline code is not correctly parsed, you will see a message
\newcommand{\rinline}[1]{SOMETHING WORNG WITH knitr}




\begin{document}
\section*{Detective work}

\hspace{12 pt} The Dade County spouse assault experiment was executed
excellently as a randomized field experiment [Pate and Hamilton
(1992), hereafter PH]. Since randomization does not justify logistic
regression as presented in PH, we would like to recover the experimental
data and conduct another anlysis that is justified by the randomization. 


In order to conduct comparion on rates of re-abuse,
between the treatment group (arrestees) and the control
group, as well as the same rate comparison within the two subgroups
(unemployed/employed subjects), we need the numbers in the following table:

\begin{center}
\begin{tabular}{l|l|l|l}
                    & no arrest            & arrest                & \\ \hline
unemployed & {$n_{00}/N_{00}$} & {$n_{01}/N_{01}$} & {$n_{0.}/N_{0.}$}\\ \hline
employed     & {$n_{10}/N_{10}$} & {$n_{11}/N_{11}$} & {$n_{1.}/N_{1.}$}\\ \hline
                    & {$n_{.0}/N_{.0}$}   & {$n_{.1}/N_{.1}$}  & {$n_{..}/N_{..}$}\\ 
\end{tabular}
\end{center}


\subsubsection*{1. Subject counts}
\hspace{12 pt} The total subject counts can be directly derived from the codebook and
the dataset file \textit{part\_607}. After basic data cleaning, we
re-group the data based on column \textit{act} into a new two-level
column \textit{trass}, i.e. treament assgined, following the
\textit{intention-to-treat} principle. So with cross-tabulation, we
get all the \textit{N}'s for the table.

\begin{knitrout}
\definecolor{shadecolor}{rgb}{0.969, 0.969, 0.969}\color{fgcolor}\begin{kframe}
\begin{verbatim}
##      trass
## sempl   0   1
##     0 139 124
##     1 306 338
\end{verbatim}
\end{kframe}
\end{knitrout}


A simple sanity check on the matching of the data to the PH paper is
done with the unemployment rate among the subjects, which is 
28.9967\% from the data and
agrees with $29\%$ in the PH paper.


\subsubsection*{2. Re-abusing subject counts}
\hspace{12 pt} Without a proper linked dataset on the number of
re-abusing subjects, we could only recover the \textit{n}'s from the
PH Figure 1 percentages. We round the calculated numbers to the nearest integers to
reflect the fact that they are real human counts.

\begin{knitrout}
\definecolor{shadecolor}{rgb}{0.969, 0.969, 0.969}\color{fgcolor}\begin{kframe}
\begin{verbatim}
##      trass
## sempl  0  1
##     0 10 21
##     1 38 21
\end{verbatim}
\end{kframe}
\end{knitrout}


A similar sanity check would be the re-abusing rates. The rate of
re-abuse among arrestees is 9.0909\%,
which agrees with the $9.0\%$ in PH paper. Also, the re-abusing rate of
non-arrestees is 10.7865\%,  
slightly bigger than the PH $10.6\%$ due to rounding.


\subsubsection*{}
So, the final result of our data recovery work gives the table below.

\begin{center}
\begin{tabular}{l|l|l|l}
                    & no arrest     & arrest          & \\ \hline
unemployed & {$10/139$} & {$21/124$} & {$31/263$}\\ \hline
employed     & {$38/306$} & {$21/338$} & {$59/644$}\\ \hline
                    & {$48/445$} & {$42/462$} & {$90/907$}\\ 
\end{tabular}
\end{center}




\newpage
\section*{Statistical work}
\hspace{12 pt} Based on the recovered data, we would like to
re-evaluate the three conclusions drawn in the PH paper based their
logistic regression analyses. We would like to compare the relevant
observed rates in the above table with serveral methods, including
Fisher's exact test and two-sample test of equal binomial proportions.


In more details, the two-sample test of equal binomial proportions use
the test statistic
\begin{center}
$\displaystyle z=\frac{p_1 - p_2}{\sqrt{p(1-p)(\frac{1}{n_1}+\frac{1}{n_2})}}$
\end{center}
where $p_1=n_1/N_1$, $p_2=n_2/N_2$, $p=(n_1+n_2)/(N_1+N_2)$, and
follows normal distribution check.\newline




\subsubsection*{1. re-abuse $\sim$ arrest | employed}
\hspace{12 pt} We would like to assess the conclusion that "Among
employed suspects, arrest had a statistically significant deterrent
effect on the occurrence of a subsequent assault", with the hypothesis
testing given below.


Among employed suspects,


\hspace{12 pt} \textit{H0: the occurrence rate of a subsequent assault
  is the same between the arrestees and non-arrestees} \newline
\vspace{2 pt}
\hspace{24 pt} \textit{HA: the re-abuse rate is higher among the
  non-arrestees than that of the arrestees} \newline


Employed suspects:

\begin{knitrout}
\definecolor{shadecolor}{rgb}{0.969, 0.969, 0.969}\color{fgcolor}\begin{kframe}
\begin{verbatim}
##           non-arrest arrest
## reabuse           38     21
## non-recid        268    317
## 
## 	Fisher's Exact Test for Count Data
## 
## data:  cl1 
## p-value = 0.004715
## alternative hypothesis: true odds ratio is greater than 1 
## 95 percent confidence interval:
##  1.297   Inf 
## sample estimates:
## odds ratio 
##      2.138
\end{verbatim}
\end{kframe}
\end{knitrout}


Using the test statistic given above, the $p$-value for the two-sample
binomial test is 0.0032, which is
close to that of Fisher's exact test. Both of them suggest that we can
reject the null hypothesis and the drawn conclusion is acceptable.



\subsubsection*{2. re-abuse $\sim$ arrest | unemployed}
\hspace{12 pt} We would like to assess the conclusion that "Among
unemployed suspects, ... significant increases in subsequent assault
were associated with arrest", with the hypothesis
testing given below.


Among unemployed suspects,


\hspace{12 pt} \textit{H0: the occurrence rate of a subsequent assault
  is the same between the arrestees and non-arrestees} \newline
\vspace{2 pt}
\hspace{24 pt} \textit{HA: the re-abuse rate is lower among the
  non-arrestees than that of the arrestees} \newline

\newpage
Unemployed suspects:
\begin{knitrout}
\definecolor{shadecolor}{rgb}{0.969, 0.969, 0.969}\color{fgcolor}\begin{kframe}
\begin{verbatim}
##           non-arrest arrest
## reabuse           10     21
## non-recid        129    103
## 
## 	Fisher's Exact Test for Count Data
## 
## data:  cl2 
## p-value = 0.01184
## alternative hypothesis: true odds ratio is less than 1 
## 95 percent confidence interval:
##  0.0000 0.7901 
## sample estimates:
## odds ratio 
##     0.3816
\end{verbatim}
\end{kframe}
\end{knitrout}



Similarly, using the test statistic given above, the $p$-value for the two-sample
binomial test is 0.0072, which is also
close to that of Fisher's exact test. Moreover, both of them suggest that we can
reject the null hypothesis and the drawn conclusion in PH is acceptable.



\subsubsection*{3. re-abuse $\sim$ arrest}
\hspace{12 pt} We would like to assess the final conclusion that "[Among
all suspects, there is] no statistically significant effect of arrest
on the occurrence of a subsequent spouse assault", with the hypothesis
testing given below.


Among all suspects,


\hspace{12 pt} \textit{H0: the occurrence rate of a subsequent assault
  is the same between the arrestees and non-arrestees} \newline
\vspace{2 pt}
\hspace{24 pt} \textit{HA: the re-abuse rate is different among the
  non-arrestees from the arrestees} \newline


All suspects:

\begin{knitrout}
\definecolor{shadecolor}{rgb}{0.969, 0.969, 0.969}\color{fgcolor}\begin{kframe}
\begin{verbatim}
##           non-arrest arrest
## reabuse           48     42
## non-recid        397    420
## 
## 	Fisher's Exact Test for Count Data
## 
## data:  cl3 
## p-value = 0.4372
## alternative hypothesis: true odds ratio is not equal to 1 
## 95 percent confidence interval:
##  0.7635 1.9199 
## sample estimates:
## odds ratio 
##      1.209
\end{verbatim}
\end{kframe}
\end{knitrout}


Using the test statistic given above, the $p$-value for the two-sample
binomial test is 0.4017, which is
close to that of Fisher's exact test. Both of them suggest that we can
NOT reject the null hypothesis and the PH conclusion is acceptable.


\newpage
\subsubsection*{Randomization}
\hspace{12 pt} About the Fisher's exact test, the $p$-value is
computed conditionally on the row and column margins being
constant. In our application, this assumption suggest that the total
number of observed re-abusers would not change if there had been a
different randomization outcome in the Dade County experiment. 

In the Neyman model of experiment, the $Y_i^T$ and $Y_i^C$ is
unknown but deterministic, with only the assignment $X_i$ being
random. The total number of the observed re-abusers, in theory, can
change in the overall population under a different random
assignment. And if the employment rate is fixed, so will the
subgroups. 
However, in reality, this Fisher's exact test assumption could
be really close to the truth if the $Y^T$'s and $Y^C$'s share similar
success rate. This is verified by the similar $p$-values calculated by
the two sample Binomial test.


On the other hand, the binomial test
concerns independent Bernoulli trials. In our experiment, the re-abuse
outcomes are not considered as random coin flips, but only the
treatment assginment is pure random. 

However, the randomization can justify that the two distributions
(treatment and control) are independent regardless of their underlying
details of the distributions. Even though not exactly binomial, the
distributions can be approximated by binomial distributions, generating
$p$-values close to those in the Fisher's exact test, which casts
hypergeometric distribution on the 2-by-2 table.


\end{document}
